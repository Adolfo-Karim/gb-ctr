\chapter{MBC5 mapper chip}

The majority of games for Game Boy Color use the MBC5 chip. MBC5 supports ROM
sizes up to 64 Mbit (512 banks of \hex{4000} bytes), and RAM sizes up to 1 Mbit
(16 banks of \hex{2000} bytes). The information in this section is based on my
MBC5 research, and The Cycle-Accurate Game Boy Docs \cite{tcagbd}.

\section{MBC5 registers}

\begin{register}[H]
  \caption{\hexrange{0000}{1FFF} - RAMG - MBC5 RAM gate register}
  {
    \ttfamily
    \begin{tabularx}{\textwidth}{|X|X|X|X|X|X|X|X|}
      \hline
      W-0 & W-0 & W-0 & W-0 & W-0 & W-0 & W-0 & W-0 \\
      \hline
      \multicolumn{8}{|c|}{RAMG<7:0>} \\
      \hline
      bit 7 & 6 & 5 & 4 & 3 & 2 & 1 & bit 0 \\
      \hline
    \end{tabularx}
  }

  \begin{description}[leftmargin=5em, style=nextline]
    \item[bit 7-0]
      \textbf{RAMG<7:0>}: RAM gate register \\
      \bin{00001010}= enable access to cartridge RAM \\
      All other values disable access to cartridge RAM
  \end{description}
\end{register}

The 8-bit MBC5 RAMG register works in a similar manner as MBC1 RAMG, but it is
a full 8-bit register so upper bits matter when writing to it. Only
\bin{00001010} enables RAM access, and all other values (including
\bin{10001010} for example) disable access to RAM.

When RAM access is disabled, all writes to the external RAM area
\hexrange{A000}{BFFF} are ignored, and reads return undefined values. Pan Docs
recommends disabling RAM when it's not being accessed to protect the contents
\cite{pandocs}.

\begin{speculation}
  We don't know the physical implementation of RAMG, but it's certainly
  possible that the \bin{00001010} bit pattern check is done at write time and the
  register actually consists of just a single bit.
\end{speculation}

\begin{register}[H]
  \caption{\hexrange{2000}{2FFF} - ROMB0 - MBC5 lower ROM bank register}

  {
    \ttfamily
    \begin{tabularx}{\textwidth}{|X|X|X|X|X|X|X|X|}
      \hline
      W-0 & W-0 & W-0 & W-0 & W-0 & W-0 & W-0 & W-1 \\
      \hline
      \multicolumn{8}{|c|}{ROMB0<7:0>} \\
      \hline
      bit 7 & 6 & 5 & 4 & 3 & 2 & 1 & bit 0 \\
      \hline
    \end{tabularx}
  }

  \begin{description}[leftmargin=5em, style=nextline]
    \item[bit 7-0]
      \textbf{ROMB0<7:0>}: Lower ROM bank register \\
  \end{description}
\end{register}

The 8-bit ROMB0 register is used as the lower 8 bits of the ROM bank number
when the CPU accesses the \hexrange{4000}{7FFF} memory area.

\begin{register}[H]
  \caption{\hexrange{3000}{3FFF} - ROMB1 - MBC5 upper ROM bank register}

  {
    \ttfamily
    \begin{tabularx}{\textwidth}{|X|X|X|X|X|X|X|X|}
      \hline
      U & U & U & U & U & U & U & W-0 \\
      \hline
      \cellcolor{LightGray} & \cellcolor{LightGray} & \cellcolor{LightGray} & \cellcolor{LightGray} & \cellcolor{LightGray} & \cellcolor{LightGray} & \cellcolor{LightGray} & ROMB1 \\
      \hline
      bit 7 & 6 & 5 & 4 & 3 & 2 & 1 & bit 0 \\
      \hline
    \end{tabularx}
  }

  \begin{description}[leftmargin=5em, style=nextline]
    \item[bit 7-1]
      \textbf{Unimplemented}: Ignored during writes
    \item[bit 0]
      \textbf{ROMB1}: Upper ROM bank register \\
  \end{description}
\end{register}

The 1-bit ROMB1 register is used as the most significant bit (bit 9) of the ROM
bank number when the CPU accesses the \hexrange{4000}{7FFF} memory area.

\begin{register}[H]
  \caption{\hexrange{4000}{5FFF} - RAMB - MBC5 RAM bank register}

  {
    \ttfamily
    \begin{tabularx}{\textwidth}{|X|X|X|X|X|X|X|X|}
      \hline
      U & U & U & U & W-0 & W-0 & W-0 & W-0 \\
      \hline
      \cellcolor{LightGray} & \cellcolor{LightGray} & \cellcolor{LightGray} & \cellcolor{LightGray} & \multicolumn{4}{c|}{RAMB<3:0>} \\
      \hline
      bit 7 & 6 & 5 & 4 & 3 & 2 & 1 & bit 0 \\
      \hline
    \end{tabularx}
  }

  \begin{description}[leftmargin=5em, style=nextline]
    \item[bit 7-4]
      \textbf{Unimplemented}: Ignored during writes
    \item[bit 3-0]
      \textbf{RAMB<3:0>}: RAM bank register \\
  \end{description}
\end{register}

The 4-bit RAMB register is used as the RAM bank number when the CPU accesses
the \hexrange{A000}{BFFF} memory area.
