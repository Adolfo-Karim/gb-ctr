%!TEX program = xelatex
\providecommand{\main}{.} 

\newcount\draft\draft=1
\newcommand{\Revision}{}

\newcommand{\Revision}{@REVISION@}
\if @DRAFT@0
  \newcount\draft\draft=0
\else
  \newcount\draft\draft=1
\fi


\ifnum \draft=1
  \documentclass[a4paper, draft, oneside]{memoir}
\else
  \documentclass[a4paper, oneside]{memoir}
\fi

\usepackage[final]{graphicx}
\usepackage[english]{babel}
\usepackage{fontspec}
\usepackage[T1]{fontenc}
\usepackage{subcaption}
\usepackage{charter}
\usepackage{mathpazo}
\usepackage[ttdefault=true]{AnonymousPro}
\usepackage{tabularx}
\usepackage{pdflscape}
\usepackage{enumitem}
\usepackage{ccicons}
\usepackage[svgnames,table]{xcolor}
\usepackage{tikz}
\usepackage{tikz-timing}
\usepackage{wrapfig}
\usepackage{float}
\usepackage[cache=false]{minted}
\usepackage{enumitem}
\usepackage[many]{tcolorbox}
\usepackage{fontawesome}
\usepackage{hyperref}
\usepackage[all]{hypcap}
\usepackage{hyphenat}
\usepackage{subfiles}

\usetikztiminglibrary{either}

% aliases to support old fontawesome package versions
\providecommand\faArrowCircleDown{\faCircleArrowDown}
\providecommand\faArrowCircleUp{\faCircleArrowUp}
\providecommand\faArrowCircleRight{\faCircleArrowRight}
\providecommand\faArrowCircleLeft{\faCircleArrowLeft}
\providecommand\faExclamationCircle{\faExclamationSign}
\providecommand\faInfoCircle{\faInfoSign}

\usemintedstyle{autumn}

\graphicspath{{\main/images/}}

\bibliographystyle{plain}

\title{Game Boy: Complete Technical Reference}
\author{gekkio\\ \url{https://gekkio.fi}}
\renewcommand{\maketitlehookd}{
  \begin{center}
    Revision \Revision

    \ifdraftdoc
      DRAFT!
    \fi

    \vfill

    \href{http://creativecommons.org/licenses/by-sa/4.0/}{\Huge \ccbysa}

    This work is licensed under a \href{http://creativecommons.org/licenses/by-sa/4.0/}{Creative Commons Attribution-ShareAlike 4.0 International License}.
  \end{center}
}

\ifdraftdoc
  \makeevenfoot{plain}{}{\thepage}{\textit{DRAFT! \Revision}}
  \makeoddfoot{plain}{\textit{DRAFT! \Revision}}{\thepage}{}
  \makeevenfoot{headings}{}{}{\textit{DRAFT! \Revision}}
  \makeoddfoot{headings}{\textit{DRAFT! \Revision}}{}{}
\fi

\hypersetup{final,unicode=true,pdfborder={0 0 0},bookmarksnumbered=true,pdfpagemode=UseOutlines,pdfauthor=gekkio,pdftitle=\thetitle}

\setlrmarginsandblock{2cm}{2cm}{*}
\setulmarginsandblock{2cm}{2cm}{*}
\checkandfixthelayout

\newtcolorbox{speculation}
{colframe=SteelBlue,colback=Azure,title=\faPuzzlePiece,fonttitle=\small}

\newtcolorbox{caveat}
{colframe=Crimson,colback=MistyRose,title=\faInfoCircle,fonttitle=\small}

\newtcolorbox{warning}
{colframe=Gold,colback=LemonChiffon,title=\faExclamationCircle,fonttitle=\small}

\floatstyle{plaintop}
\newfloat{register}{h}{lor}[chapter]
\floatname{register}{Register}

\newcommand{\bit}[1]{\texttt{#1}}
\newcommand{\bin}[1]{\texttt{0b#1}}
\newcommand{\hex}[1]{\texttt{0x#1}}
\newcommand{\hexrange}[2]{\texttt{0x#1\hyp{}0x#2}}

\newcolumntype{L}{>{\raggedright\arraybackslash}X}
\newcolumntype{C}{>{\centering\arraybackslash}X}
\newcolumntype{R}{>{\raggedleft\arraybackslash}X}

\setcounter{tocdepth}{4}

\begin{document}

\hypersetup{pageanchor=false}

\begin{titlingpage}
  \calccentering{\unitlength}
  \setlength{\droptitle}{80pt}
  \begin{adjustwidth*}{\unitlength}{-\unitlength}
    \maketitle
  \end{adjustwidth*}
\end{titlingpage}

\hypersetup{pageanchor=true}

\chapter*{Preface}
\addcontentsline{toc}{chapter}{Preface}
\phantomsection

\begin{caveat}
  \Huge
  IMPORTANT: This document focuses at the moment on 1st and 2nd generation
  devices (models before the Game Boy Color), and some hardware details are
  very different in later generations.

  \bigskip

  Be very careful if you make assumptions about later generation devices based
  on this document!
\end{caveat}

\chapter*{How to read this document}
\addcontentsline{toc}{chapter}{How to read this document}
\phantomsection

\begin{speculation}
  This is something that hasn't been verified, but would make a lot of sense.
\end{speculation}

\begin{caveat}
  This explains some caveat about this documentation that you should know.
\end{caveat}

\begin{warning}
  This is a warning about something.
\end{warning}

\section{Formatting of numbers}

When a single bit is discussed in isolation, the value looks like this: \bit{0}, \bit{1}.

Binary numbers are prefixed with \bin{} like this: \bin{0101101}, \bin{11011}, \bin{00000000}. Values are prefixed with zeroes when necessary, so the total number of digits always matches the number of digits in the value.

Hexadecimal numbers are prefixed with \hex{} like this: \hex{1234}, \hex{DEADBEEF}, \hex{FF04}. Values are prefixed with zeroes when necessary, so the total number of characters always matches the number of nibbles in the value.

Examples:

\vspace{0.5cm}

\begin{tabular}{l l l l}
              & 4-bit      & 8-bit          & 16-bit                 \\
  \hline
  Binary      & \bin{0101} & \bin{10100101} & \bin{0000101010100101} \\
  Hexadecimal & \hex{5}    & \hex{A5}       & \hex{0AA5}
\end{tabular}

\section{Register definitions}

\begin{register}[H]
  \caption{\hex{1234} - This is a hardware register definition}
  {
    \ttfamily
    \begin{tabularx}{\linewidth}{|C|C|C|C|C|C|C|C|}
      \hline
      R/W-0                            & R/W-1                   & U-1                              & R-0  & R-1                      & R-x & W-1 & U-0   \\
      \hline
      \multicolumn{2}{|c|}{VALUE<1:0>} & \cellcolor{LightGray} - & \multicolumn{3}{c|}{BIGVAL<7:5>} & FLAG & \cellcolor{LightGray} - \\
      \hline
      bit 7                            & 6                       & 5                                & 4    & 3                        & 2   & 1   & bit 0 \\
      \hline
    \end{tabularx}{\parfillskip=0pt\par}
  }

  \medskip
  \textbf{Top row legend:}
  \begin{description}[leftmargin=5em, style=nextline]
    \item[R]
      Bit can be read.
    \item[W]
      Bit can be written. If the bit cannot be read, reading returns a constant
      value defined in the bit list of the register in question.
    \item[U]
      Unimplemented bit. Writing has no effect, and reading returns a constant
      value defined in the bit list of the register in question.
    \item[-n]
      Value after system reset: \bit{0}, \bit{1}, or x.
    \item[\bit{1}]
      Bit is set.
    \item[\bit{0}]
      Bit is cleared.
    \item[x]
      Bit is unknown (e.g. depends on external things such as user input).
  \end{description}

  \medskip
  \textbf{Middle row legend:}

  {
    \ttfamily
    \begin{tabularx}{0.5\linewidth}{|L|C|}
      \hline
      VALUE<1:0>              & \rmfamily Bits 1 and 0 of VALUE  \\
      \hline
      \cellcolor{LightGray} - & \rmfamily Unimplemented bit      \\
      \hline
      BIGVAL<7:5>             & \rmfamily Bits 7, 6, 5 of BIGVAL \\
      \hline
      FLAG                    & \rmfamily Single-bit value FLAG  \\
      \hline
    \end{tabularx}
  }

  \vspace{3mm}
  \textbf{In this example:}
  \begin{itemize}
    \item{After system reset, VALUE is \bin{01}, BIGVAL is either \bin{010} or \bin{011}, FLAG is \bin{1}.}
    \item{Bits 5 and 0 are unimplemented. Bit 5 always returns \bit{1}, and bit 0 always returns \bit{0}.}
    \item{Both bits of VALUE can be read and written. When this register is written, bit 7 of the written value goes to bit 1 of VALUE.}
    \item{FLAG can only be written to, so reads return a value that is defined elsewhere.}
    \item{BIGVAL cannot be written to. Only bits 5-7 of BIGVAL are defined here, so look elsewhere for the low bits 0-4.}
  \end{itemize}
\end{register}

\clearpage

\tableofcontents

\part{Sharp SM83 CPU core}

\subfile{chapter/cpu/instruction-set}

\part{Game Boy SoC peripherals and features}

\subfile{chapter/peripherals/boot-rom}
\subfile{chapter/peripherals/dma}
\subfile{chapter/peripherals/ppu}
\subfile{chapter/peripherals/p1}
\subfile{chapter/peripherals/serial}

\part{Game Boy game cartridges}

\subfile{chapter/cartridges/mbc1}
\subfile{chapter/cartridges/mbc2}
\subfile{chapter/cartridges/mbc3}
\subfile{chapter/cartridges/mbc30}
\subfile{chapter/cartridges/mbc5}
\subfile{chapter/cartridges/mbc6}
\subfile{chapter/cartridges/mbc7}
\subfile{chapter/cartridges/huc1}
\subfile{chapter/cartridges/huc3}
\subfile{chapter/cartridges/mmm01}
\subfile{chapter/cartridges/tama5}

\part*{Appendices}
\addcontentsline{toc}{part}{Appendices}
\phantomsection

\begin{appendices}
\subfile{appendix/opcode-tables}
\subfile{appendix/memory-map}
\subfile{appendix/external-bus}
\subfile{appendix/pinouts}
\end{appendices}

\bibliography{gbctr}

\end{document}
