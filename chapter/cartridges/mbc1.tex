%!TEX root = ../../gbctr.tex
%!TEX program = xelatex
\providecommand{\main}{../..}
\documentclass[\main/gbctr.tex]{subfiles}
\begin{document}

\chapter{MBC1 mapper chip}

The majority of games for the original Game Boy use the MBC1 chip. MBC1
supports ROM sizes up to 16 Mbit (128 banks of \hex{4000} bytes) and RAM
sizes up to 256 Kbit (4 banks of \hex{2000} bytes). The information in
this section is based on my MBC1 research, Tauwasser's research notes
\cite{tauwasser_mbc1}, and Pan Docs \cite{pandocs}.

\section{MBC1 registers}

\begin{caveat}
  These registers don't have any standard names and are usually referred to
  using their address ranges or purposes instead. This document uses names to
  clarify which register is meant when referring to one.
\end{caveat}

The MBC1 chip includes four registers that affect the behaviour of the chip.
Of the cartridge bus address signals, only A13-A15 are connected to the MBC, so
lower address bits don't matter when the CPU is accessing the MBC and all
registers are effectively mapped to address ranges instead of single addresses.
All registers are smaller than 8 bits, and unused bits are simply ignored
during writes. The registers are not directly readable.

\begin{register}[H]
  \caption{\hexrange{0000}{1FFF} - RAMG - MBC1 RAM gate register}
  {
    \ttfamily
    \begin{tabularx}{\linewidth}{|X|X|X|X|X|X|X|X|}
      \hline
      U                     & U                     & U                     & U                     & W-0                                    & W-0 & W-0 & W-0   \\
      \hline
      \cellcolor{LightGray} & \cellcolor{LightGray} & \cellcolor{LightGray} & \cellcolor{LightGray} & \multicolumn{4}{c|}{RAMG<3:0>} \\
      \hline
      bit 7                 & 6                     & 5                     & 4                     & 3                                      & 2   & 1   & bit 0 \\
      \hline
    \end{tabularx}{\parfillskip=0pt\par}
  }

  \begin{description}[leftmargin=5em, style=nextline]
    \item[bit 7-4]
      \textbf{Unimplemented}: Ignored during writes
    \item[bit 3-0]
      \textbf{RAMG<3:0>}: RAM gate register\\
      \bin{1010}= enable access to cartridge RAM\\
      All other values disable access to cartridge RAM
  \end{description}
\end{register}

The RAMG register is used to enable access to the cartridge SRAM if one exists
on the cartridge circuit board. RAM access is disabled by default but can be
enabled by writing to the \hexrange{0000}{1FFF} address range a value with the
bit pattern \bin{1010} in the lower nibble. Upper bits don't matter, but any
other bit pattern in the lower nibble disables access to RAM.

When RAM access is disabled, all writes to the external RAM area
\hexrange{A000}{BFFF} are ignored, and reads return undefined values. Pan Docs
recommends disabling RAM when it's not being accessed to protect the contents
\cite{pandocs}.

\begin{speculation}
  We don't know the physical implementation of RAMG, but it's certainly
  possible that the \bin{1010} bit pattern check is done at write time and the
  register actually consists of just a single bit.
\end{speculation}

\begin{register}[H]
  \caption{\hexrange{2000}{3FFF} - BANK1 - MBC1 bank register 1}

  {
    \ttfamily
    \begin{tabularx}{\linewidth}{|X|X|X|X|X|X|X|X|}
      \hline
      U                     & U                     & U                     & W-0                                  & W-0 & W-0 & W-0 & W-1   \\
      \hline
      \cellcolor{LightGray} & \cellcolor{LightGray} & \cellcolor{LightGray} & \multicolumn{5}{c|}{BANK1<4:0>} \\
      \hline
      bit 7                 & 6                     & 5                     & 4                                    & 3   & 2   & 1   & bit 0 \\
      \hline
    \end{tabularx}{\parfillskip=0pt\par}
  }

  \begin{description}[leftmargin=5em, style=nextline]
    \item[bit 7-5]
      \textbf{Unimplemented}: Ignored during writes
    \item[bit 4-0]
      \textbf{BANK1<4:0>}: Bank register 1\\
      Never contains the value \bin{00000}.\\
      If \bin{00000} is written, the resulting value will be \bin{00001} instead.
  \end{description}
\end{register}

The 5-bit BANK1 register is used as the lower 5 bits of the ROM bank number
when the CPU accesses the \hexrange{4000}{7FFF} memory area.

MBC1 doesn't allow the BANK1 register to contain zero (bit pattern
\bin{00000}), so the initial value at reset is \bin{00001} and attempting to
write \bin{00000} will write \bin{00001} instead. This makes it impossible to
read banks \hex{00}, \hex{20}, \hex{40} and \hex{60} from the
\hexrange{4000}{7FFF} memory area, because those bank numbers have \bin{00000}
in the lower bits. Due to the zero value adjustment, requesting any of these
banks actually requests the next bank (e.g.  \hex{21} instead of \hex{20}).

\begin{register}[H]
  \caption{\hexrange{4000}{5FFF} - BANK2 - MBC1 bank register 2}

  {
    \ttfamily
    \begin{tabularx}{\linewidth}{|X|X|X|X|X|X|X|X|}
      \hline
      U                     & U                     & U                     & U                     & U                     & U                     & W-0                                  & W-0   \\
      \hline
      \cellcolor{LightGray} & \cellcolor{LightGray} & \cellcolor{LightGray} & \cellcolor{LightGray} & \cellcolor{LightGray} & \cellcolor{LightGray} & \multicolumn{2}{c|}{BANK2<1:0>} \\
      \hline
      bit 7                 & 6                     & 5                     & 4                     & 3                     & 2                     & 1                                    & bit 0 \\
      \hline
    \end{tabularx}{\parfillskip=0pt\par}
  }

  \begin{description}[leftmargin=5em, style=nextline]
    \item[bit 7-2]
      \textbf{Unimplemented}: Ignored during writes
    \item[bit 1-0]
      \textbf{BANK2<1:0>}: Bank register 2
  \end{description}
\end{register}

The 2-bit BANK2 register can be used as the upper bits of the ROM bank number,
or as the 2-bit RAM bank number. Unlike BANK1, BANK2 doesn't disallow zero, so
all 2-bit values are possible.

\begin{register}
  \caption{\hexrange{6000}{7FFF} - MODE - MBC1 mode register}

  {
    \ttfamily
    \begin{tabularx}{\linewidth}{|X|X|X|X|X|X|X|X|}
      \hline
      U                     & U                     & U                     & U                           & U & U & U & W-0   \\
      \hline
      \cellcolor{LightGray} & \cellcolor{LightGray} & \cellcolor{LightGray} & \cellcolor{LightGray} & \cellcolor{LightGray} & \cellcolor{LightGray} & \cellcolor{LightGray} & MODE \\
      \hline
      bit 7                 & 6                     & 5                     & 4                              & 3   & 2   & 1   & bit 0 \\
      \hline
    \end{tabularx}{\parfillskip=0pt\par}
  }

  \begin{description}[leftmargin=5em, style=nextline]
    \item[bit 7-1]
      \textbf{Unimplemented}: Ignored during writes
    \item[bit 0]
      \textbf{MODE}: Mode register\\
      \bin{1}= BANK2 affects accesses to \hexrange{0000}{3FFF}, \hexrange{4000}{7FFF}, \hexrange{A000}{BFFF}\\
      \bin{0}= BANK2 affects only accesses to \hexrange{4000}{7FFF}
  \end{description}
\end{register}

The MODE register determines how the BANK2 register value is used during memory
accesses.

\begin{warning}
  Most documentation, including Pan Docs \cite{pandocs}, calls value \bin{0}
  ROM banking mode, and value \bin{1} RAM banking mode. This terminology
  reflects the common use cases, but "RAM banking" is slightly misleading because
  value \bin{1} also affects ROM reads in multicart cartridges and cartridges
  that have a 8 or 16 Mbit ROM chip.
\end{warning}

\section{ROM in the \hexrange{0000}{7FFF} area}

In MBC1 cartridges, the A0-A13 cartridge bus signals are connected directly to
the corresponding ROM pins, and the remaining ROM pins (A14-A20) are controlled
by the MBC1. These remaining pins form the ROM bank number.

When the \hexrange{0000}{3FFF} address range is accessed, the effective bank
number depends on the MODE register. In MODE \bin{0} the bank number is always
0, but in MODE \bin{1} it's formed by shifting the BANK2 register value left by
5 bits.

When the \hexrange{4000}{7FFF} addess range is accessed, the effective bank
number is always a combination of BANK1 and BANK2 register values.

If the cartridge ROM is smaller than 16 Mbit, there are less ROM address pins
to connect to and therefore some bank number bits are ignored. For example, 4
Mbit ROMs only need a 5-bit bank number, so the BANK2 register value is always
ignored because those bits are simply not connected to the ROM.

\begin{table}[H]
  \caption{Mapping of physical ROM address bits in MBC1 carts}
  \centering
  \begin{tabular}{|l|c|c|c|}
    \hline
    & \multicolumn{3}{c|}{ROM address bits} \\
    Accessed address & \multicolumn{2}{c|}{Bank number} & Address within bank \\
    \hline
    & 20-19 & 18-14 & 13-0 \\
    \hline
    \hexrange{0000}{3FFF}, MODE = \bin{0} & \bin{00} & \bin{00000} & A<13:0> \\
    \hline
    \hexrange{0000}{3FFF}, MODE = \bin{1} & BANK2 & \bin{00000} & A<13:0> \\
    \hline
    \hexrange{4000}{7FFF} & BANK2 & BANK1 & A<13:0> \\
    \hline
  \end{tabular}
\end{table}

\subsection{ROM banking example 1}

Let's assume we have previously written \hex{12} to the BANK1 register and
\bin{01} to the BANK2 register. The effective bank number during ROM reads
depends on which address range we read and on the value of the MODE register:

\begin{description}[style=nextline]
  \item[Value of the BANK1 register]
  {
    \ttfamily
    0b\colorbox{blue!30}{10010}
  }
  \item[Value of the BANK2 register]
  {
    \ttfamily
    0b\colorbox{red!30}{01}
  }
  \item[Effective ROM bank number (reading \hexrange{4000}{7FFF})]
  {
    \ttfamily
    0b\colorbox{red!30}{01}\colorbox{blue!30}{10010} (= 50 = \hex{32})
  }
  \item[Effective ROM bank number (reading \hexrange{0000}{3FFF}, MODE = \bin{0})]
  {
    \ttfamily
    0b\colorbox{gray!10}{00}\colorbox{gray!10}{00000} (= 0 = \hex{00})
  }
  \item[Effective ROM bank number (reading \hexrange{0000}{3FFF}, MODE = \bin{1})]
  {
    \ttfamily
    0b\colorbox{red!30}{01}\colorbox{gray!10}{00000} (= 32 = \hex{20})
  }
\end{description}

\subsection{ROM banking example 2}

Let's assume we have previously requested ROM bank number 68, MBC1 mode is
\bin{0}, and we are now reading a byte from \hex{72A7}. The actual
physical ROM address that will be read is going to be \hex{1132A7} and is
constructed in the following way:

\begin{description}[leftmargin=15em,style=nextline]
  \item[Value of the BANK1 register]
  {
    \ttfamily
    0b\colorbox{blue!30}{00100}
  }
  \item[Value of the BANK2 register]
  {
    \ttfamily
    0b\colorbox{red!30}{10}
  }
  \item[ROM bank number]
  {
    \ttfamily
    0b\colorbox{red!30}{10}\colorbox{blue!30}{00100} (= 68 = \hex{44})
  }
  \item[Address being read]
  {
    \ttfamily
    0b\colorbox{gray!10}{01}\colorbox{green!30}{11 0010 1010 0111} (= \hex{72A7})
  }
  \item[Actual physical ROM address]
  {
    \ttfamily
    0b\colorbox{red!30}{1 0}\colorbox{blue!30}{001 00}\colorbox{green!30}{11 0010 1010 0111} (= \hex{1132A7})
  }
\end{description}

\section{RAM in the \hexrange{A000}{BFFF} area}

Some MBC1 carts include SRAM, which is mapped to the \hexrange{A000}{BFFF}
area. If no RAM is present, or RAM is not enabled with the RAMG register, all
reads return undefined values and writes have no effect.

On boards that have RAM, the A0-A12 cartridge bus signals are connected
directly to the corresponding RAM pins, and pins A13-A14 are controlled by the
MBC1. Most of the time the RAM size is 64 Kbit, which corresponds to a single
bank of \hex{2000} bytes. With larger RAM sizes the BANK2 register value can be
used for RAM banking to provide the two high address bits.

In MODE \bin{0} the BANK2 register value is not used, so the first RAM bank is
always mapped to the \hexrange{A000}{BFFF} area. In MODE \bin{1} the BANK2
register value is used as the bank number.

\begin{table}[H]
  \caption{Mapping of physical RAM address bits in MBC1 carts}
  \centering
  \begin{tabular}{|l|c|c|}
    \hline
    & \multicolumn{2}{c|}{RAM address bits} \\
    Accessed address & Bank number & Address within bank \\
    \hline
    & 14-13 & 12-0 \\
    \hline
    \hexrange{A000}{BFFF}, MODE = \bin{0} & \bin{00} & A<12:0> \\
    \hline
    \hexrange{A000}{BFFF}, MODE = \bin{1} & BANK2 & A<12:0> \\
    \hline
  \end{tabular}
\end{table}

\subsection{RAM banking example 1}

Let's assume we have previously written \bin{10} to the BANK2 register, MODE is
\bin{1}, RAMG is \bin{1010} and we are now reading a byte from \hex{B123}.  The
actual physical RAM address that will be read is going to be \hex{5123} and is
constructed in the following way:

\begin{description}[leftmargin=15em,style=nextline]
  \item[Value of the BANK2 register]
  {
    \ttfamily
    0b\colorbox{red!30}{10}
  }
  \item[Address being read]
  {
    \ttfamily
    0b\colorbox{gray!10}{101}\colorbox{green!30}{1 0001 0010 0011} (= \hex{B123})
  }
  \item[Actual physical RAM address]
  {
    \ttfamily
    0b\colorbox{red!30}{10}\colorbox{green!30}{1 0001 0010 0011} (= \hex{5123})
  }
\end{description}

\section{MBC1 multicarts ("MBC1M")}

MBC1 is also used in a couple of "multicart" cartridges, which include more
than one game on the same cartridge. These cartridges use the same regular MBC1
chip, but the circuit board is wired a bit differently. This alternative wiring
is sometimes called "MBC1M", but technically the mapper chip is the same. All
known MBC1 multicarts use 8 Mbit ROMs, so there's no definitive wiring for
other ROM sizes.

In MBC1 multicarts bit 4 of the BANK1 register is not physically connected to
anything, so it's skipped. This means that the bank number is actually a 6-bit
number. In all known MBC1 multicarts the games reserve 16 banks each, so BANK2
can actually be considered "game number", while BANK1 is the internal bank
number within the selected game. At reset BANK2 is \bin{00}, and the "game" in
this slot is actually a game selection menu. The menu code selects MODE \bin{1}
and writes the game number to BANK2 once the user selects a game.

From a ROM banking point of view, multicarts simply skip bit 4 of the BANK1
register, but otherwise the behaviour is the same. MODE \bin{1} guarantees that
all ROM accesses, including accesses to \hexrange{0000}{3FFF}, use the BANK2
register value.

\begin{table}[H]
  \caption{Mapping of physical ROM address bits in MBC1 multicarts}
  \centering
  \begin{tabular}{|l|c|c|c|}
    \hline
    & \multicolumn{3}{c|}{ROM address bits} \\
    Accessed address & \multicolumn{2}{c|}{Bank number} & Address within bank \\
    \hline
    & 19-18 & 17-14 & 13-0 \\
    \hline
    \hexrange{0000}{3FFF}, MODE = \bin{0} & \bin{00} & \bin{0000} & A<13:0> \\
    \hline
    \hexrange{0000}{3FFF}, MODE = \bin{1} & BANK2 & \bin{0000} & A<13:0> \\
    \hline
    \hexrange{4000}{7FFF} & BANK2 & BANK1<3:0> & A<13:0> \\
    \hline
  \end{tabular}
\end{table}

\subsection{ROM banking example 1}

Let's assume we have previously requested "game number" 3 (= \bin{11})
and ROM bank number 29 (= \hex{1D}), MBC1 mode is \bin{1}, and we are
now reading a byte from \hex{6C15}. The actual physical ROM address that
will be read is going to be \hex{F6C15} and is constructed in the
following way:

\begin{description}[leftmargin=15em,style=nextline]
  \item[Value of the BANK1 register]
  {
    \ttfamily
    0b\colorbox{gray!10}{1}\colorbox{blue!30}{1101}
  }
  \item[Value of the BANK2 register]
  {
    \ttfamily
    0b\colorbox{red!30}{11}
  }
  \item[ROM bank number]
  {
    \ttfamily
    0b\colorbox{red!30}{11}\colorbox{blue!30}{1101} (= 61 = \hex{3D})
  }
  \item[Address being read]
  {
    \ttfamily
    0b\colorbox{gray!10}{01}\colorbox{green!30}{10 1100 0001 0101} (= \hex{6C15})
  }
  \item[Actual physical ROM address]
  {
    \ttfamily
    0b\colorbox{red!30}{11}\colorbox{blue!30}{11 01}\colorbox{green!30}{10 1100 0001 0101} (= \hex{F6C15})
  }
\end{description}

\subsection{Detecting multicarts}

MBC1 multicarts are not detectable by simply looking at the ROM header, because
the ROM type value is just one of the normal MBC1 values. However, detection is
possible by going through BANK2 values and looking at "bank 0" of each
multicart game and doing some heuristics based on the header data. All the
included games, including the game selection menu, have proper header data.
One example of a good heuristic is logo data verification.

So, if you have a 8 Mbit cart with MBC1, first assume that it's a multicart and
bank numbers are 6-bit values. Set BANK1 to zero and loop through the four
possible BANK2 values while checking the data at \hexrange{0104}{0133}. In
other words, check logo data starting from physical ROM locations \hex{00104},
\hex{40104}, \hex{80104}, and \hex{C0104}. If proper logo data exists with most
of the BANK2 values, the cart is most likely a multicart. Note that multicarts
can just have two actual games, so one of the locations might not have the
header data in place.

\section{Dumping MBC1 carts}

MBC1 cartridge dumping is fairly straightforward with the right hardware. The
total number of banks is read from the header, and each bank is read one byte
at a time. However, BANK1 register zero-adjustment and multicart cartridges
need to be considered in ROM dumping code.

Banks \hex{20}, \hex{40} and \hex{60} can only be read from the
\hexrange{0000}{3FFF} memory area and only when MODE register value is \bin{1}.
Using MODE \bin{1} has no undesirable effects when doing ROM dumping, so using
it at all times is recommended for simplicity.

Multicarts should be detected using the logo check described earlier, and if a
multicart is detected, BANK1 should be considered a 4-bit register in the
dumping code.

\begin{listing}[H]
  \inputminted[frame=lines]{python}{code-snippets/mbc1_rom_dump.py}
  \caption{Python pseudo-code for MBC1 ROM dumping}
\end{listing}

\end{document}
