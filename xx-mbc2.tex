\chapter{MBC2 mapper chip}

MBC2 supports ROM sizes up to 2 Mbit (16 banks of \hex{4000} bytes) and
includes an internal 512x4 bit RAM array, which is its unique feature. The
information in this section is based on my MBC2 research, Tauwasser's research
notes \cite{tauwasser_mbc2}, and Pan Docs \cite{pandocs}.

\begin{speculation}
  MBC1 is strictly more powerful than MBC2, because it supports more ROM and
  RAM. This raises a very important question: why does MBC2 exist? It's
  possible that Nintendo tried to integrate a small amount of RAM on the MBC
  chip for cost reasons, but it seems that this didn't work out very well since
  all later MBCs revert this design decision and use separate RAM chips.
\end{speculation}

\section{MBC2 registers}

\begin{caveat}
  These registers don't have any standard names and are usually referred to
  using their address ranges or purposes instead. This document uses names to
  clarify which register is meant when referring to one.
\end{caveat}

The MBC2 chip includes two registers that affect the behaviour of the chip.

\begin{register}[H]
  \caption{RAMG - MBC2 RAM gate register}
  {
    \ttfamily
    \begin{tabularx}{\textwidth}{|X|X|X|X|X|X|X|X|}
      \hline
      U                     & U                     & U                     & U                     & W-0                                    & W-0 & W-0 & W-0   \\
      \hline
      \cellcolor{LightGray} & \cellcolor{LightGray} & \cellcolor{LightGray} & \cellcolor{LightGray} & \multicolumn{4}{c|}{RAMG<3:0>} \\
      \hline
      bit 7                 & 6                     & 5                     & 4                     & 3                                      & 2   & 1   & bit 0 \\
      \hline
    \end{tabularx}
  }

  \begin{description}[leftmargin=5em, style=nextline]
    \item[bit 7-4]
      \textbf{Unimplemented}: Ignored during writes
    \item[bit 3-0]
      \textbf{RAMG<3:0>}: RAM gate register \\
      \bin{1010}= enable access to chip RAM \\
      All other values disable access to chip RAM
  \end{description}
\end{register}

\begin{register}[H]
  \caption{ROMB - MBC2 ROM bank register}
  {
    \ttfamily
    \begin{tabularx}{\textwidth}{|X|X|X|X|X|X|X|X|}
      \hline
      U                     & U                     & U                     & U                     & W-0                                    & W-0 & W-0 & W-1   \\
      \hline
      \cellcolor{LightGray} & \cellcolor{LightGray} & \cellcolor{LightGray} & \cellcolor{LightGray} & \multicolumn{4}{c|}{ROMB<3:0>} \\
      \hline
      bit 7                 & 6                     & 5                     & 4                     & 3                                      & 2   & 1   & bit 0 \\
      \hline
    \end{tabularx}
  }

  \begin{description}[leftmargin=5em, style=nextline]
    \item[bit 7-4]
      \textbf{Unimplemented}: Ignored during writes
    \item[bit 3-0]
      \textbf{ROMB<3:0>}: ROM bank register \\
      Never contains the value \bin{0000}. \\
      If \bin{0000} is written, the resulting value will be \bin{0001} instead.
  \end{description}
\end{register}
